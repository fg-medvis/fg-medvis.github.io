\documentclass[german,11pt]{artikel1}
\usepackage[ngerman]{babel}

\usepackage{bera}
\renewcommand{\familydefault}{\sfdefault}

\usepackage{xcolor}
\definecolor{tuftsblue}{rgb}{0.96, 0.96, 0.96}
\usepackage[colorlinks=false,urlcolor=black,linkcolor=black]{hyperref}
\hypersetup{
  pdftitle    = {Karl-Heinz Hoehne Award},
  pdfsubject  = {Fragebogen 2023},
  pdfauthor   = {BEWERBER}
  allbordercolors=1 1 1,
  pdfborderstyle={/S/U/W 1}
  }
\usepackage[latin1]{inputenc}
\usepackage{anysize}

\usepackage{fancyhdr}
\headheight = 45 pt
\pagestyle{fancy}
\lhead{{\huge medvis award}\\\large Karl-Heinz-H\"ohne-Preis}
\rhead{\large 2023}
\lfoot{}
\rfoot{\tiny SD/2023}
\cfoot{\thepage}

\renewcommand{\arraystretch}{1.5}
\setlength{\parindent}{0pt}


\begin{document}

\title{Fragebogen}
\author{GI Fachgruppe Visual Computing in Biology and Medicine}
\date{\today}
\maketitle
\thispagestyle{fancy}


Im Folgenden sind die zu beantwortenden Fragen zur Einreichung einer Arbeit f�r den Wettbewerb zum 
Karl-Heinz-H�hne-Preis 2023 f�r medizinische Visualisierung (medvis award) aufgelistet.
Der Preis wird von der und von der {\em Fachgruppe Visual Computing in Biology and Medicine}, Teil der {\em Gesellschaft f�r Informatik},
ausgelobt.
Informationen zum Preis k�nnen unter \href{https://www.fg-medvis.de/appaward.html}{https://www.fg-medvis.de/appaward.html} abgerufen werden.


\section{Worin liegt die Relevanz der Arbeit? Was ist der Bezug zu medizinischer Ausbildung, Diagnostik oder Therapie?}

\section{Was ist das Originelle an der Arbeit?}


\section{Welcher Fortschritt in Bezug auf das Problem ist erreicht worden?}

%\clearpage

\section{Welche Arbeiten (Ver�ffentlichungen) sind am ehesten mit der eingereichten vergleichbar?}

\section{Inwiefern ist die Arbeit von Benutzern getestet worden?}

\section{Einreichung bei anderen Wettbewerben}

Folgende Arbeit ist bereits bei einem welchem anderen Wettbewerb eingereicht oder ausgezeichnet worden:

% keine bzw. komplette Angaben

\section{Zur Person}

\begin{tabular}{lp{11cm}}
	Name & \\
	Vorname & \\
	Anschrift & \\
	& \\
% Str., Hausnr, PLZ, Ort
	Telephonnummer & \\
	E-Mail & \\
 	Aktuelle Arbeitsstelle & \\
	& \\
% (Bitte nennen Sie Titel der Universit�t, Fachbereich bzw. Firma, Abteilung und Ort.)
\end{tabular}

\vspace{1.5cm}
\begin{tabbing}
\hspace{6cm} \= \hspace{7cm}\kill
\underline{\hspace{5cm}} \>\underline{\hspace{7cm}}\\
\small Ort, Datum \> \small Unterschrift\\
\end{tabbing}


\end{document}


